\documentclass[
  man,
  floatsintext,
  longtable,
  nolmodern,
  notxfonts,
  notimes,
  colorlinks=true,linkcolor=blue,citecolor=blue,urlcolor=blue]{apa7}

\usepackage{amsmath}
\usepackage{amssymb}




\RequirePackage{longtable}
\RequirePackage{threeparttablex}

\makeatletter
\renewcommand{\paragraph}{\@startsection{paragraph}{4}{\parindent}%
	{0\baselineskip \@plus 0.2ex \@minus 0.2ex}%
	{-.5em}%
	{\normalfont\normalsize\bfseries\typesectitle}}

\renewcommand{\subparagraph}[1]{\@startsection{subparagraph}{5}{0.5em}%
	{0\baselineskip \@plus 0.2ex \@minus 0.2ex}%
	{-\z@\relax}%
	{\normalfont\normalsize\bfseries\itshape\hspace{\parindent}{#1}\textit{\addperi}}{\relax}}
\makeatother




\usepackage{longtable, booktabs, multirow, multicol, colortbl, hhline, caption, array, float, xpatch}
\setcounter{topnumber}{2}
\setcounter{bottomnumber}{2}
\setcounter{totalnumber}{4}
\renewcommand{\topfraction}{0.85}
\renewcommand{\bottomfraction}{0.85}
\renewcommand{\textfraction}{0.15}
\renewcommand{\floatpagefraction}{0.7}

\usepackage{tcolorbox}
\tcbuselibrary{listings,theorems, breakable, skins}
\usepackage{fontawesome5}

\definecolor{quarto-callout-color}{HTML}{909090}
\definecolor{quarto-callout-note-color}{HTML}{0758E5}
\definecolor{quarto-callout-important-color}{HTML}{CC1914}
\definecolor{quarto-callout-warning-color}{HTML}{EB9113}
\definecolor{quarto-callout-tip-color}{HTML}{00A047}
\definecolor{quarto-callout-caution-color}{HTML}{FC5300}
\definecolor{quarto-callout-color-frame}{HTML}{ACACAC}
\definecolor{quarto-callout-note-color-frame}{HTML}{4582EC}
\definecolor{quarto-callout-important-color-frame}{HTML}{D9534F}
\definecolor{quarto-callout-warning-color-frame}{HTML}{F0AD4E}
\definecolor{quarto-callout-tip-color-frame}{HTML}{02B875}
\definecolor{quarto-callout-caution-color-frame}{HTML}{FD7E14}

%\newlength\Oldarrayrulewidth
%\newlength\Oldtabcolsep


\usepackage{hyperref}




\providecommand{\tightlist}{%
  \setlength{\itemsep}{0pt}\setlength{\parskip}{0pt}}
\usepackage{longtable,booktabs,array}
\usepackage{calc} % for calculating minipage widths
% Correct order of tables after \paragraph or \subparagraph
\usepackage{etoolbox}
\makeatletter
\patchcmd\longtable{\par}{\if@noskipsec\mbox{}\fi\par}{}{}
\makeatother
% Allow footnotes in longtable head/foot
\IfFileExists{footnotehyper.sty}{\usepackage{footnotehyper}}{\usepackage{footnote}}
\makesavenoteenv{longtable}

\usepackage{graphicx}
\makeatletter
\newsavebox\pandoc@box
\newcommand*\pandocbounded[1]{% scales image to fit in text height/width
  \sbox\pandoc@box{#1}%
  \Gscale@div\@tempa{\textheight}{\dimexpr\ht\pandoc@box+\dp\pandoc@box\relax}%
  \Gscale@div\@tempb{\linewidth}{\wd\pandoc@box}%
  \ifdim\@tempb\p@<\@tempa\p@\let\@tempa\@tempb\fi% select the smaller of both
  \ifdim\@tempa\p@<\p@\scalebox{\@tempa}{\usebox\pandoc@box}%
  \else\usebox{\pandoc@box}%
  \fi%
}
% Set default figure placement to htbp
\def\fps@figure{htbp}
\makeatother


% definitions for citeproc citations
\NewDocumentCommand\citeproctext{}{}
\NewDocumentCommand\citeproc{mm}{%
  \begingroup\def\citeproctext{#2}\cite{#1}\endgroup}
\makeatletter
 % allow citations to break across lines
 \let\@cite@ofmt\@firstofone
 % avoid brackets around text for \cite:
 \def\@biblabel#1{}
 \def\@cite#1#2{{#1\if@tempswa , #2\fi}}
\makeatother
\newlength{\cslhangindent}
\setlength{\cslhangindent}{1.5em}
\newlength{\csllabelwidth}
\setlength{\csllabelwidth}{3em}
\newenvironment{CSLReferences}[2] % #1 hanging-indent, #2 entry-spacing
 {\begin{list}{}{%
  \setlength{\itemindent}{0pt}
  \setlength{\leftmargin}{0pt}
  \setlength{\parsep}{0pt}
  % turn on hanging indent if param 1 is 1
  \ifodd #1
   \setlength{\leftmargin}{\cslhangindent}
   \setlength{\itemindent}{-1\cslhangindent}
  \fi
  % set entry spacing
  \setlength{\itemsep}{#2\baselineskip}}}
 {\end{list}}
\usepackage{calc}
\newcommand{\CSLBlock}[1]{\hfill\break\parbox[t]{\linewidth}{\strut\ignorespaces#1\strut}}
\newcommand{\CSLLeftMargin}[1]{\parbox[t]{\csllabelwidth}{\strut#1\strut}}
\newcommand{\CSLRightInline}[1]{\parbox[t]{\linewidth - \csllabelwidth}{\strut#1\strut}}
\newcommand{\CSLIndent}[1]{\hspace{\cslhangindent}#1}


\usepackage[nolongtablepatch]{lineno}
\linenumbers



\usepackage{newtx}

\defaultfontfeatures{Scale=MatchLowercase}
\defaultfontfeatures[\rmfamily]{Ligatures=TeX,Scale=1}





\title{eyetools: an R package for simplified analysis of eye data}


\shorttitle{eyetools: eye data analysis}


\usepackage{etoolbox}








\authorsnames{Tom Beesley,Matthew Ivory}





\affiliation{
{Lancaster University}}




\leftheader{Beesley and Ivory}



\abstract{Abstract goes here}

\keywords{eye-tracking; fixations; saccades; areas-of-interest}

\authornote{\par{\addORCIDlink{Tom Beesley}{0000-0003-2836-2743}} 

\par{       }
\par{Correspondence concerning this article should be addressed to Tom
Beesley, Lancaster University, Department of Psychology, Lancaster
University, UK, LA1 4YD, UK, Email: t.beesley@lancaster.ac.uk}
}

\makeatletter
\let\endoldlt\endlongtable
\def\endlongtable{
\hline
\endoldlt
}
\makeatother

\urlstyle{same}



\makeatletter
\@ifpackageloaded{caption}{}{\usepackage{caption}}
\AtBeginDocument{%
\ifdefined\contentsname
  \renewcommand*\contentsname{Table of contents}
\else
  \newcommand\contentsname{Table of contents}
\fi
\ifdefined\listfigurename
  \renewcommand*\listfigurename{List of Figures}
\else
  \newcommand\listfigurename{List of Figures}
\fi
\ifdefined\listtablename
  \renewcommand*\listtablename{List of Tables}
\else
  \newcommand\listtablename{List of Tables}
\fi
\ifdefined\figurename
  \renewcommand*\figurename{Figure}
\else
  \newcommand\figurename{Figure}
\fi
\ifdefined\tablename
  \renewcommand*\tablename{Table}
\else
  \newcommand\tablename{Table}
\fi
}
\@ifpackageloaded{float}{}{\usepackage{float}}
\floatstyle{ruled}
\@ifundefined{c@chapter}{\newfloat{codelisting}{h}{lop}}{\newfloat{codelisting}{h}{lop}[chapter]}
\floatname{codelisting}{Listing}
\newcommand*\listoflistings{\listof{codelisting}{List of Listings}}
\makeatother
\makeatletter
\makeatother
\makeatletter
\@ifpackageloaded{caption}{}{\usepackage{caption}}
\@ifpackageloaded{subcaption}{}{\usepackage{subcaption}}
\makeatother

% From https://tex.stackexchange.com/a/645996/211326
%%% apa7 doesn't want to add appendix section titles in the toc
%%% let's make it do it
\makeatletter
\xpatchcmd{\appendix}
  {\par}
  {\addcontentsline{toc}{section}{\@currentlabelname}\par}
  {}{}
\makeatother

%% Disable longtable counter
%% https://tex.stackexchange.com/a/248395/211326

\usepackage{etoolbox}

\makeatletter
\patchcmd{\LT@caption}
  {\bgroup}
  {\bgroup\global\LTpatch@captiontrue}
  {}{}
\patchcmd{\longtable}
  {\par}
  {\par\global\LTpatch@captionfalse}
  {}{}
\apptocmd{\endlongtable}
  {\ifLTpatch@caption\else\addtocounter{table}{-1}\fi}
  {}{}
\newif\ifLTpatch@caption
\makeatother

\begin{document}

\maketitle


\setcounter{secnumdepth}{-\maxdimen} % remove section numbering

\setlength\LTleft{0pt}

\resetlinenumber[1]

\section{Introduction}\label{introduction}

Eye tracking is now an established and widely used technique in the
behavioural sciences. Perhaps the scientific discipline with the most
invested interest in eye-data is Psychology, where eye-tracking systems
are now commonplace in almost all university departments. Beyond
academic institutions, eye-tracking continues to be a useful tool in
understanding consumer behaviour, user-interface design, and in various
forms of entertainment.

By recording the movement of an individual's gaze during research
studies, researchers can quantify where and how long individual's look
at particular regions of space (usually with a focus on stimuli
presented on a 2D screen, but also within 3D space). Eye tracking
provides a rich stream of continuous data and therefore can offer
powerful insights into real-time cognitive processing. Such data allow
researchers to inspect the interplay of cognitive processes such as
attention, memory, and decision making, with high temporal precision
(\citeproc{ref-duchowski2017eye}{Duchowski \& Duchowski, 2017}).

While there are abundant uses and benefits of collecting eye-movement
data in psychology experiments, the continual stream of recording can
lead to an overwhelming amount of raw data: modern eye-trackers can
record data at 1000 Hz and above, which results in 3.6 million rows of
data per hour. The provision of suitable computational software for data
reduction and processing is an important part of eye-tracking research.
The companies behind eye-tracking devices offer licensed software that
will perform many of the necessary steps for eye-data analysis. However,
there are several disadvantages to using such proprietary software in a
research context. Firstly, the software will typically have an ongoing
(annual) license cost for continual use. Secondly, the algorithms
driving the operations within such software are not readily available
for inspection. Both of these important constraints mean that the use of
proprietary analysis software will lead to a failure to meet the basic
open-science principle of analysis reproduction, for example as set out
by the UK Reprodicibility Network: ``We expect researchers to\ldots{}
make their research methods, software, outputs and data open, and
available at the earliest possible point\ldots The reproducibility of
both research methods and research results \ldots is critical to
research in certain contexts, particularly in the experimental sciences
with a quantitative focus\ldots{}''

In the current article we introduce a new toolkit for eye-data
processing and analysis called ``\emph{eyetools}'', which takes the form
of an R package. R packages (like R itself) are free to use without
licence and are therefore available for any user across the world. The
package provides a (growing) number of functions that provide an
efficient and effective means to conduct basic eye-data analysis.
\emph{eyetools} is built with academic researchers in the psychological
sciences in mind, though there is no reason why the package would not be
effective more generally. The functions within the package reflect steps
in a comprehensive analysis workflow, taking the user from initial
handling of raw eye data, to summarising data for each period of a
procedure, to the visualisation of the data in plots. We hope that the
functions are simple enough to mean that the package is easy to use for
researchers who are unfamiliar with working with eye data. It should
also appeal to researchers accustomed to working with eye data in other
environments who wish to transfer to working in R.

\emph{eyetools} is, of course, not the only package in R that allows
users to work with eye data. A recent assessment of available packages
on CRAN identified six other packages that offer relevant functions for
the analysis of eye data. \textbf{eyeTrackr}, \textbf{eyelinker}, and
\textbf{eyelinkReader}, all offer functionality for data only from
experiments that have used `EyeLink' trackers (S-R Research). In
contrast, eyetools provides functions that are hardware-agnostic,
relying on a format of data that can be achieved from any data source.
The \textbf{eyeRead} package is designed for the analysis of eye data
from reading exercises. The \textbf{emov} package offers a limited set
of functions and is primarily designed for fixation detection, using the
same dispersion algorithm used in eyetools. Finally,
\textbf{eyetrackingR} is perhaps the most comprehensive alternative
package available on CRAN. It has functions for cleaning data and
various plotting functions, including analysis over time. It does not
feature algorithms for detection of fixations or saccades, instead
working with raw data {[}is this true?{]}.

{[}insert table of features here{]}

In this tutorial we demonstrate the pipeline of analysis functions
within eyetools. The package has been designed to be simple to use by
someone with basic knowledge of data handling and analysis in R. It
should appeal to researchers who are working with raw eye data for the
first time, as well as those accustomed to working with eye data in
other environments who wish to transfer to working in R.

This tutorial is separated into five distinct sections. In the first
section, we briefly describe the basic methodology of collecting eye
data in general, and in regard to the specific dataset we use to
illustrate all the functionality of the eyetools package. The second
section covers the process for getting data from an eye tracker into an
eyetools-friendly format. The third section introduces the foundational
functions of the eyetools package, from repairing and smoothing eye
data, to calculating fixations and saccades, and detecting time spent in
Areas of Interest (AOIs). The fourth section takes the processed data,
and applies basic analysis techniques commonplace in eye data research.
In the fifth and final section, we reflect on the benefits of the
eyetools package, including contributions to open science practices,
reproducibility, and providing clarity to eye data analysis.

\section{Data Collection}\label{data-collection}

First describe basic paradigms for collecting eye data. Also purpose
etc. @tom

Then describe the specifics of the dataset we are using - I presume this
is the HCL dataset in full? Should make mention of the fact that the
workflow can be done either with the full dataset, or the two
participant dataset provided in package.

\section{Converting Raw Data}\label{converting-raw-data}

@matthew, @tom

Owing to the vast range of eye tracking hardware available, eyetools
does not offer much in the way of converting raw data into the eyetools
format of participant ID, trial number, timestamp, x, and y coordinates.
In this section, we cover the stages of transforming the data from a
TOBII eye tracker.

\texttt{hdf5\_to\_csv()}

When you have data that is in a binocular format, that is you have a set
of coordinates for each eye, it needs to be converted into single x and
y coordinates for both eyes combined. Using eyetools, this can be done
in one of two ways, either taking an average of the coordinattes from
the two eyes, or by choosing the eye with the fewest missing samples is
used to represent the data. An averaging of the two coordinates sets is
the typical method of combining binocular data, and can be done using
the \texttt{combine\_eyes()} function in eyetools. This returns a
flattened list of participant data that has x and y variables in place
of the left\_* and right\_* variables.

\section{Working with eyetools}\label{working-with-eyetools}

In the last section, we finished with the data in a format that holds
participant ID, trial number, a timestamp, along with x and y
coordinates. This is the format expected by eyetools when working with
multi-participant data, however if you some reason you are working with
a single participant then the participant ID column is superfluous and
can be dropped. This basic data format of ID, trial, time, x, and y
ensures that eyetools is applicable to a variety of eye data sources and
does not depend on specific eye trackers being used.

\subsubsection{Counterbalanced designs}\label{counterbalanced-designs}

Many psychology experiments will position stimuli on the screen in a
counterbalanced fashion. For example, in the example data we are using,
there are two stimuli, with one of these appearing on the left and one
on the right. In our design, one of the cue stimuli is a ``target'' and
one is a ``distractor'', and the experiment counterbalances whether
these are positioned on the left or right across trials.

Eyetools has a built in function which allows us to transform the x (or
y) values of the stimuli to take into account a counterbalancing
variable: \texttt{conditional\_transform()}. This function currently
allows for a single-dimensional flip across either the horizontal or
vertical midline. It can be used on raw data or fixation data. It
requires the spatial coordinates (x, y) and a specification of the
counterbalancing variable. The result is a normalised set of data, in
which the x (and/or y) position is consistent across counterbalanced
conditions (e.g., in our example, we can transform the data so that the
target cue is always on the left). This transformation is especially
useful for future visualisations and calculation of time on areas of
interest. Note that \texttt{conditional\_transform()} is another
function that does not discriminate between multi-participant and
single-participant data and so no participant\_ID parameter is required.
To transform the data, we require knowledge of where predictive cues
were presented. Using this, \texttt{conditional\_transform()} can align
data across the x or y midline, depending on how stimuli were presented.
In the experimental design used in our study, cues were presented either
on the left or the right, so by applying
\texttt{conditional\_transform()}, all the predictive cues in the
dataset are recorded on the same side.

\subsection{Repairing missing data and smoothing
data}\label{repairing-missing-data-and-smoothing-data}

Despite researcher's best efforts and hopes, participants are likely to
blink during data collection, resulting in observations where no data is
present for where the eyes would be looking. To mitigate this issue, the
\texttt{interpolate()} function estimates the path taken by the eyes
based upon the eye coordinates before and after the missing data. There
are two methods for estimating the path, linear interpolation
(``approx'', the default setting) and cubic spline (``spline''). The
default method of linear interpolation replaces missing values with a
line of constant slope and evenly spaced coordinates reaching the
existing data. The cubic spline method applies piecewise cubic functions
to enable a curve to be calculated as opposed to a line between points.

When using \texttt{interpolate()}, a report can be requested so that a
researcher can measure how much missing data has been replaced. This
parameter changes the output format of the function, and returns a list
of both the data and the report. The report can be accessed easily using
the following code:

\begin{verbatim}
  pNum missing_perc_before missing_perc_after
1  118          0.02314313        0.015838577
2  119          0.01214128        0.005402579
\end{verbatim}

As shown, not all missing data has been replaced, this is because when
gaps are larger than a given size they are kept as missing data due to
it being unreasonable to try to estimate the path taken by the eye. The
amount of missing data that will be estimated can be changed through the
maxgap parameter.

Once missing data has been fixed, a common step is to smooth the eye
data to remove particularly jerky eye movements. To do this,
\texttt{smoother()} reduces the noise in the data by applying a moving
averaging function. The degree of smoothing can be specified, as well as
having a plot generated for random trials to observe how well the
smoothed data fits the raw data.

\pandocbounded{\includegraphics[keepaspectratio]{BRM_ms_files/figure-pdf/unnamed-chunk-6-1.pdf}}

\subsection{Fixations}\label{fixations}

Once the data has been repaired and smoothed, a core step in eye data
analysis is to identify fixations
(\citeproc{ref-salvucci2000identifying}{Salvucci \& Goldberg, 2000}),
defined as when the gaze stops in a specific location for a given amount
of time. When the eyes are moving between these fixations, they are
considered to be saccades. Subsequently, data can be split into these
two groups, fixations and saccades. In the eyetools package, there are
two fixation algorithms offered; the first algorithm,
\texttt{fixation\_dispersion()} employs a dispersion-based approach that
uses spatial and temporal data to determine fixations. By using a
maximum dispersion range, the algorithm looks for sufficient periods of
time that the eye gaze remains within this range and once this range is
exceed, this is termed as a fixation. The second algorithm,
\texttt{fixation\_VTI()} takes advantage of the idea that data is either
a fixation or a saccade and employs a velocity-threshold approach. It
identifies data where the eye is moving at a minimum velocity and
excludes this, before applying a dispersion check to ensure that the eye
does not drift during the fixation period. If the range is broken, a new
fixation is determined. Saccades must be of a given length to be
removed, otherwise they are considered as micro-saccades
{[}@CITATION\_NEEDED\_HERE?{]}.

Once fixations have been calculated, they can be used in conjunction
with Areas of Interest (AOIs) to determine the sequence in which the eye
enters and exits these areas, as well as the time spent in these
regions. When referring to AOIs, these often refer to the cues presented
and the outcome object. In our example, the two cues at the top of the
screen are the cues, and the outcome is at the bottom. We can define
these areas in a separate dataframe object by giving the centrepoint of
the AOI in x, y coordinates along with the width and height (if the AOIs
are rectangular) or just the radius (if circular).

In combination with the fixation data, the AOI information can be used
to determine the sequence of AOI entries using the \texttt{AOI\_seq()}
function. This fucntion checks whether a fixation is detected within an
AOI, and if not, it is dropped from the output, and then provides a list
of the sequence of AOI entries, along with start and end timestamps, and
the duration.

Time spent in AOIs can also be calculated from fixations or raw data
using the \texttt{AOI\_time()} function available. This calculates the
time spent in each AOI in each trial, based on the data type given, in
our case fixation data.

If choosing to work with the raw data, there is also the option of using
\texttt{AOI\_time\_binned()} which allows for the trials to be split
into bins of a given length, and the time spent in AOIs calculated as a
result.

\section{Analysing eye data}\label{analysing-eye-data}

@tom

\section{Discussion}\label{discussion}

In the present tutorial, we began by identifying the current gap in
available tools for working with eye data in open-science pipelines. We
then provided an overview of the general data collection process
required for eye tracking research, before detailing the conversion of
raw eye data into a useable eyetools format. We then covered the entire
processing pipeline using functions available in the eyetools package
that included the repairing and normalising the data, and the detection
of events such as fixations, saccades, and AOI entries.
@SOMETHING\_ON\_THE\_ANALYSIS\_GOES\_HERE.

From a practical perspective, this tutorial offers a step-by-step
walkthrough for handling eye data using R for open-science, reproducible
purposes. It provides a pipeline that can be relied upon by novices
looking to work with eye data, as well as offering new functions and
tools for experienced researchers. By enabling the processing and
analysis of data in a single R environment it also helps to speed up
data analysis.

\subsection{Advantages of Open-Source
Tools}\label{advantages-of-open-source-tools}

eyetools offers an open-source toolset that holds no hidden nor
proprietary functionality.

\section{Data Availability}\label{data-availability}

The data required for reproducing this tutorial is available at: @URL. A
condensed version of the dataset (starting with the
\texttt{combine\_eyes()} function) is a dataset in the eyetools package
called HCL.

\section{Code Availability}\label{code-availability}

The code used in this tutorial is available in the reproducible
manuscript file available at:

\section{References}\label{references}

\phantomsection\label{refs}
\begin{CSLReferences}{1}{0}
\bibitem[\citeproctext]{ref-duchowski2017eye}
Duchowski, A. T., \& Duchowski, A. T. (2017). \emph{Eye tracking
methodology: Theory and practice}. Springer.

\bibitem[\citeproctext]{ref-salvucci2000identifying}
Salvucci, D. D., \& Goldberg, J. H. (2000). Identifying fixations and
saccades in eye-tracking protocols. \emph{Proceedings of the 2000
Symposium on Eye Tracking Research \& Applications}, 71--78.

\end{CSLReferences}






\end{document}
