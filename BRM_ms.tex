\documentclass[
  man,
  floatsintext,
  longtable,
  nolmodern,
  notxfonts,
  notimes,
  colorlinks=true,linkcolor=blue,citecolor=blue,urlcolor=blue]{apa7}

\usepackage{amsmath}
\usepackage{amssymb}




\RequirePackage{longtable}
\RequirePackage{threeparttablex}

\makeatletter
\renewcommand{\paragraph}{\@startsection{paragraph}{4}{\parindent}%
	{0\baselineskip \@plus 0.2ex \@minus 0.2ex}%
	{-.5em}%
	{\normalfont\normalsize\bfseries\typesectitle}}

\renewcommand{\subparagraph}[1]{\@startsection{subparagraph}{5}{0.5em}%
	{0\baselineskip \@plus 0.2ex \@minus 0.2ex}%
	{-\z@\relax}%
	{\normalfont\normalsize\bfseries\itshape\hspace{\parindent}{#1}\textit{\addperi}}{\relax}}
\makeatother




\usepackage{longtable, booktabs, multirow, multicol, colortbl, hhline, caption, array, float, xpatch}
\setcounter{topnumber}{2}
\setcounter{bottomnumber}{2}
\setcounter{totalnumber}{4}
\renewcommand{\topfraction}{0.85}
\renewcommand{\bottomfraction}{0.85}
\renewcommand{\textfraction}{0.15}
\renewcommand{\floatpagefraction}{0.7}

\usepackage{tcolorbox}
\tcbuselibrary{listings,theorems, breakable, skins}
\usepackage{fontawesome5}

\definecolor{quarto-callout-color}{HTML}{909090}
\definecolor{quarto-callout-note-color}{HTML}{0758E5}
\definecolor{quarto-callout-important-color}{HTML}{CC1914}
\definecolor{quarto-callout-warning-color}{HTML}{EB9113}
\definecolor{quarto-callout-tip-color}{HTML}{00A047}
\definecolor{quarto-callout-caution-color}{HTML}{FC5300}
\definecolor{quarto-callout-color-frame}{HTML}{ACACAC}
\definecolor{quarto-callout-note-color-frame}{HTML}{4582EC}
\definecolor{quarto-callout-important-color-frame}{HTML}{D9534F}
\definecolor{quarto-callout-warning-color-frame}{HTML}{F0AD4E}
\definecolor{quarto-callout-tip-color-frame}{HTML}{02B875}
\definecolor{quarto-callout-caution-color-frame}{HTML}{FD7E14}

%\newlength\Oldarrayrulewidth
%\newlength\Oldtabcolsep


\usepackage{hyperref}




\providecommand{\tightlist}{%
  \setlength{\itemsep}{0pt}\setlength{\parskip}{0pt}}
\usepackage{longtable,booktabs,array}
\usepackage{calc} % for calculating minipage widths
% Correct order of tables after \paragraph or \subparagraph
\usepackage{etoolbox}
\makeatletter
\patchcmd\longtable{\par}{\if@noskipsec\mbox{}\fi\par}{}{}
\makeatother
% Allow footnotes in longtable head/foot
\IfFileExists{footnotehyper.sty}{\usepackage{footnotehyper}}{\usepackage{footnote}}
\makesavenoteenv{longtable}

\usepackage{graphicx}
\makeatletter
\newsavebox\pandoc@box
\newcommand*\pandocbounded[1]{% scales image to fit in text height/width
  \sbox\pandoc@box{#1}%
  \Gscale@div\@tempa{\textheight}{\dimexpr\ht\pandoc@box+\dp\pandoc@box\relax}%
  \Gscale@div\@tempb{\linewidth}{\wd\pandoc@box}%
  \ifdim\@tempb\p@<\@tempa\p@\let\@tempa\@tempb\fi% select the smaller of both
  \ifdim\@tempa\p@<\p@\scalebox{\@tempa}{\usebox\pandoc@box}%
  \else\usebox{\pandoc@box}%
  \fi%
}
% Set default figure placement to htbp
\def\fps@figure{htbp}
\makeatother


% definitions for citeproc citations
\NewDocumentCommand\citeproctext{}{}
\NewDocumentCommand\citeproc{mm}{%
  \begingroup\def\citeproctext{#2}\cite{#1}\endgroup}
\makeatletter
 % allow citations to break across lines
 \let\@cite@ofmt\@firstofone
 % avoid brackets around text for \cite:
 \def\@biblabel#1{}
 \def\@cite#1#2{{#1\if@tempswa , #2\fi}}
\makeatother
\newlength{\cslhangindent}
\setlength{\cslhangindent}{1.5em}
\newlength{\csllabelwidth}
\setlength{\csllabelwidth}{3em}
\newenvironment{CSLReferences}[2] % #1 hanging-indent, #2 entry-spacing
 {\begin{list}{}{%
  \setlength{\itemindent}{0pt}
  \setlength{\leftmargin}{0pt}
  \setlength{\parsep}{0pt}
  % turn on hanging indent if param 1 is 1
  \ifodd #1
   \setlength{\leftmargin}{\cslhangindent}
   \setlength{\itemindent}{-1\cslhangindent}
  \fi
  % set entry spacing
  \setlength{\itemsep}{#2\baselineskip}}}
 {\end{list}}
\usepackage{calc}
\newcommand{\CSLBlock}[1]{\hfill\break\parbox[t]{\linewidth}{\strut\ignorespaces#1\strut}}
\newcommand{\CSLLeftMargin}[1]{\parbox[t]{\csllabelwidth}{\strut#1\strut}}
\newcommand{\CSLRightInline}[1]{\parbox[t]{\linewidth - \csllabelwidth}{\strut#1\strut}}
\newcommand{\CSLIndent}[1]{\hspace{\cslhangindent}#1}


\usepackage[nolongtablepatch]{lineno}
\linenumbers



\usepackage{newtx}

\defaultfontfeatures{Scale=MatchLowercase}
\defaultfontfeatures[\rmfamily]{Ligatures=TeX,Scale=1}





\title{eyetools: an R package for open-source analysis of eye data}


\shorttitle{eyetools: eye data analysis}


\usepackage{etoolbox}








\authorsnames{Tom Beesley,Matthew Ivory}





\affiliation{
{Lancaster University}}




\leftheader{Beesley and Ivory}



\abstract{Abstract goes here}

\keywords{eye-tracking; fixations; saccades; areas-of-interest}

\authornote{\par{\addORCIDlink{Tom Beesley}{0000-0003-2836-2743}} 

\par{       }
\par{Correspondence concerning this article should be addressed to Tom
Beesley, Lancaster University, Department of Psychology, Lancaster
University, UK, LA1 4YD, UK, Email: t.beesley@lancaster.ac.uk}
}

\makeatletter
\let\endoldlt\endlongtable
\def\endlongtable{
\hline
\endoldlt
}
\makeatother

\urlstyle{same}



\makeatletter
\@ifpackageloaded{caption}{}{\usepackage{caption}}
\AtBeginDocument{%
\ifdefined\contentsname
  \renewcommand*\contentsname{Table of contents}
\else
  \newcommand\contentsname{Table of contents}
\fi
\ifdefined\listfigurename
  \renewcommand*\listfigurename{List of Figures}
\else
  \newcommand\listfigurename{List of Figures}
\fi
\ifdefined\listtablename
  \renewcommand*\listtablename{List of Tables}
\else
  \newcommand\listtablename{List of Tables}
\fi
\ifdefined\figurename
  \renewcommand*\figurename{Figure}
\else
  \newcommand\figurename{Figure}
\fi
\ifdefined\tablename
  \renewcommand*\tablename{Table}
\else
  \newcommand\tablename{Table}
\fi
}
\@ifpackageloaded{float}{}{\usepackage{float}}
\floatstyle{ruled}
\@ifundefined{c@chapter}{\newfloat{codelisting}{h}{lop}}{\newfloat{codelisting}{h}{lop}[chapter]}
\floatname{codelisting}{Listing}
\newcommand*\listoflistings{\listof{codelisting}{List of Listings}}
\makeatother
\makeatletter
\makeatother
\makeatletter
\@ifpackageloaded{caption}{}{\usepackage{caption}}
\@ifpackageloaded{subcaption}{}{\usepackage{subcaption}}
\makeatother
\makeatletter
\@ifpackageloaded{fontawesome5}{}{\usepackage{fontawesome5}}
\makeatother

% From https://tex.stackexchange.com/a/645996/211326
%%% apa7 doesn't want to add appendix section titles in the toc
%%% let's make it do it
\makeatletter
\xpatchcmd{\appendix}
  {\par}
  {\addcontentsline{toc}{section}{\@currentlabelname}\par}
  {}{}
\makeatother

%% Disable longtable counter
%% https://tex.stackexchange.com/a/248395/211326

\usepackage{etoolbox}

\makeatletter
\patchcmd{\LT@caption}
  {\bgroup}
  {\bgroup\global\LTpatch@captiontrue}
  {}{}
\patchcmd{\longtable}
  {\par}
  {\par\global\LTpatch@captionfalse}
  {}{}
\apptocmd{\endlongtable}
  {\ifLTpatch@caption\else\addtocounter{table}{-1}\fi}
  {}{}
\newif\ifLTpatch@caption
\makeatother

\begin{document}

\maketitle


\setcounter{secnumdepth}{-\maxdimen} % remove section numbering

\setlength\LTleft{0pt}

\resetlinenumber[1]

Eye tracking is now an established and widely used technique in the
behavioural sciences. Perhaps the scientific discipline with the most
invested interest in eye-data is Psychology, where eye-tracking systems
are now commonplace in centres of academic research. Beyond academic
institutions, eye-tracking continues to be a useful tool in
understanding consumer behaviour, user-interface design, and in various
forms of entertainment.

By recording the movement of an individual's gaze during research
studies, researchers can quantify where and how long individual's look
at particular regions of space (usually with a focus on stimuli
presented on a 2D screen, but also within 3D space). Eye-tracking
provides a rich stream of continuous data and therefore can offer
powerful insights into real-time cognitive processing. Such data allow
researchers to inspect the interplay of cognitive processes such as
attention, memory, and decision making, with high temporal precision
(\citeproc{ref-beesley2019}{Beesley et al., 2019}).

While there are abundant uses and benefits of collecting eye-movement
data in psychology experiments, the continual stream of recording can
lead to an overwhelming amount of raw data: modern eye-trackers can
record data at 1000 Hz and above, which results in 3.6 million rows of
data per hour. The provision of suitable computational software for data
reduction and processing is an important part of eye-tracking research.
The companies behind eye-tracking devices offer licensed software that
will perform many of the necessary steps for eye-data analysis. However,
there are several disadvantages to using such proprietary software in a
research context. Firstly, the software will typically have an ongoing
license cost for continual use. Secondly, the algorithms driving the
operations within such software are not readily available for
inspection. Both of these important constraints mean that the use of
proprietary analysis software will lead to a failure to meet the basic
open-science principle of analysis reproduction, for example as set out
by the UK Reproducibility Network: ``We expect researchers to\ldots{}
make their research methods, software, outputs and data open, and
available at the earliest possible point\ldots The reproducibility of
both research methods and research results \ldots is critical to
research in certain contexts, particularly in the experimental sciences
with a quantitative focus\ldots{}''

In the current article we introduce a new toolkit for eye-data
processing and analysis called ``\emph{eyetools}'', which takes the form
of an R package. R packages (like R itself) are free to use without
licence and are therefore available for any user across the world. The
package provides a (growing) number of functions that provide an
efficient and effective means to conduct basic eye-data analysis.
\emph{eyetools} is built with academic researchers in the psychological
sciences in mind, though there is no reason why the package would not be
effective more generally. The functions within the package reflect steps
in a comprehensive analysis workflow, taking the user from initial
handling of raw eye data, to summarising data for each period of a
procedure, to the visualisation of the data in plots. We hope that the
functions are simple enough to mean that the package is easy to use for
researchers who are unfamiliar with working with eye data. It should
also appeal to researchers accustomed to working with eye data in other
environments who wish to transfer to working in R.

\emph{eyetools} is, of course, not the only package in R that allows
users to work with eye data. A recent assessment of available packages
on CRAN identified six other packages that offer relevant functions for
the analysis of eye data. \textbf{eyeTrackr}, \textbf{eyelinker}, and
\textbf{eyelinkReader}, all offer functionality for data only from
experiments that have used `EyeLink' trackers (S-R Research). In
contrast, eyetools provides functions that are hardware-agnostic,
relying on a format of data that can be achieved from any data source.
The \textbf{eyeRead} package is designed for the analysis of eye data
from reading exercises. The \textbf{emov} package offers a limited set
of functions and is primarily designed for fixation detection, using the
same dispersion method employed in eyetools. Finally,
\textbf{eyetrackingR} is perhaps the most comprehensive alternative
package available on CRAN. eyetrackingR offers a large suite of
functionality and, like eyetools, can be applied across the entire
pipeline. It has functions for cleaning data and various plotting
functions, including analysis over time. It does not feature algorithms
regarding the detection of events such as saccades or fixations. This
limits the ability to conduct more bespoke analysis steps and it means
that analysis needs to be conducted on raw data. This is disadvantageous
both in terms of computing time and in the open sharing of data (event
data are an order of magnitude smaller in size than raw data). In
comparison, eyetools enables easier data processing up to data analyses,
making core tasks in the eye data processing easier and standardised.

\begin{longtable}[]{@{}
  >{\raggedright\arraybackslash}p{(\linewidth - 12\tabcolsep) * \real{0.1429}}
  >{\centering\arraybackslash}p{(\linewidth - 12\tabcolsep) * \real{0.1429}}
  >{\centering\arraybackslash}p{(\linewidth - 12\tabcolsep) * \real{0.1429}}
  >{\centering\arraybackslash}p{(\linewidth - 12\tabcolsep) * \real{0.1429}}
  >{\centering\arraybackslash}p{(\linewidth - 12\tabcolsep) * \real{0.1429}}
  >{\centering\arraybackslash}p{(\linewidth - 12\tabcolsep) * \real{0.1429}}
  >{\centering\arraybackslash}p{(\linewidth - 12\tabcolsep) * \real{0.1429}}@{}}
\toprule\noalign{}
\begin{minipage}[b]{\linewidth}\raggedright
Package
\end{minipage} & \begin{minipage}[b]{\linewidth}\centering
Hardware-agnostic
\end{minipage} & \begin{minipage}[b]{\linewidth}\centering
Data Impor
\end{minipage} & \begin{minipage}[b]{\linewidth}\centering
Data processing
\end{minipage} & \begin{minipage}[b]{\linewidth}\centering
Identifies events
\end{minipage} & \begin{minipage}[b]{\linewidth}\centering
Plotting
\end{minipage} & \begin{minipage}[b]{\linewidth}\centering
Inferential Analysis
\end{minipage} \\
\midrule\noalign{}
\endhead
\bottomrule\noalign{}
\endlastfoot
eyetools & \faIcon{check} & \faIcon{check}* & \faIcon{check} &
\faIcon{check} & \faIcon{check} & \\
eyeTrackr & & \faIcon{check} & \faIcon{check} & \faIcon{check} & & \\
eyelinker & & \faIcon{check} & & & & \\
eyelinkReader & & \faIcon{check} & & \faIcon{check} & \faIcon{check}
& \\
eyeRead & \faIcon{check} & & \faIcon{check} & \faIcon{check}** & & \\
emov & \faIcon{check} & & \faIcon{check} & \faIcon{check} & & \\
eyetrackingR & \faIcon{check} & & \faIcon{check} & & \faIcon{check} &
\faIcon{check} \\
\end{longtable}

* for Tobii data only, ** for text reading experiments only

In this tutorial we demonstrate the pipeline of analysis functions
within \emph{eyetools}. The package has been designed to be simple to
use by someone with basic knowledge of data handling and analysis in R.
This tutorial is separated into five distinct sections. In the first
section, we briefly describe the basic methodology of collecting eye
data in general, and in regard to the specific dataset we use to
illustrate all the functionality of the \emph{eyetools} package. The
second section covers the process for getting data from an eye tracker
into an \emph{eyetools}-friendly format. The third section introduces
the foundational functions of the \emph{eyetools} package, from
repairing and smoothing eye data, to calculating fixations and saccades,
and detecting time spent in Areas of Interest (AOIs). The fourth section
takes the processed data, and applies basic analysis techniques
commonplace in eye data research. In the fifth and final section, we
reflect on the benefits of the \emph{eyetools} package, including
contributions to open science practices, reproducibility, and providing
clarity to eye data analysis.

\subsection{Installing eyetools}\label{installing-eyetools}

\emph{eyetools} is available on CRAN and can be installed with the
command \texttt{install.packages("eyetools")}. Instructions for
installing development versions can be found at the package repository:
\url{https://github.com/tombeesley/eyetools/}. Once installed, the
package can be loaded into R with the command
\texttt{library(eyetools)}.

\subsection{Preparing data for
eyetools}\label{preparing-data-for-eyetools}

Since there is a wide range of eye tracking hardware available for
researchers to use, \emph{eyetools} currently offers only a limited
number of functions for converting raw data from specific hardware. The
\texttt{hdf5\_to\_dataframe()} function is designed to work with output
from PsychoPy experiments connected to modern Tobii hardware, and will
take this default format of raw data and convert it into the simplified
raw data format required for \emph{eyetools}.

The \emph{eyetools} package has been developed primarily with the
analysis of experimental psychology data in mind. To this end, many of
the functions expect a ``trial'' variable in the data, such that the
algorithms will operate over multiple trials and produce output that
retains this trial information. Similarly, data in psychology
experiments tends to come from multiple participants, and to facilitate
analysis, a participant ID column can be included (though this isn't
necessary). This allows many functions to be run automatically across
multiple participants (rather than running the same function on each
participant's data). It is also necessary to select the relevant
``periods'' of data within the recording. It is quite typical in
psychology experiments for there to be multiple periods within a trial,
e.g., fixation; stimulus presentation; response feedback;
inter-trial-interval. \emph{eyetools} does not interpret these changes,
and so it is necessary to first select the data for the period or
periods that are of interest for analysis. Analysis on each period would
be conducted separately using the functions in \emph{eyetools}.

The starting point for the analysis pipeline is the preparation of the
raw eye data, which will consist of recorded samples from the
eye-tracker, with each row in the data reflecting a single time-stamped
recording. If the eye-tracker is set at 1000Hz, then consecutive
recordings will be 1 millisecond of time apart; at 300Hz, the recordings
are 3.33 milliseconds apart. The only requirement for the time column is
that the values reflect a consistent and increasing set of values. There
is no need to specify the sampling rate, since \emph{eyetools} functions
will calculate this automatically. \emph{eyetools} expects raw data to
have the following columns:

\begin{itemize}
\item
  x = horizontal spatial coordinate of the estimated eye position
\item
  y = vertical spatial coordinate of the estimated eye position
\item
  time = timestamp of the recording
\item
  trial = an index of the current trial in the data
\item
  pID = an index of the current participant in the data (optional)
\end{itemize}

Missing values in the x and y columns of the raw data should be
expressed as ``NA''.

For many methods of eye-tracking, binocular data will be produced. In
such cases, since the primary aim of our analyses is the estimation of
the spatial coordinates of gaze, the function \texttt{combine\_eyes()}
should be used to combine the data to form a set of monocular data. This
function takes raw data with coordinates for each eye (i.e., left\_x,
right\_x, left\_y, right\_y), and converts the data into single x and y
coordinates. By default, the function does this by taking an average of
the coordinates from the two eyes of each timestamp, but it is also
possible to select data from the eye with the lowest proportion of
missing samples. This returns a flattened list of participant data that
has x and y variables in place of the left\_* and right\_* variables.

\subsection{Repairing missing data and smoothing
data}\label{repairing-missing-data-and-smoothing-data}

Despite researcher's best efforts and hopes, participants are likely to
blink during data collection, resulting in observations where there are
NA values for the x and y coordinates. To mitigate this issue, the
\texttt{interpolate()} function estimates the gaze path taken, based
upon the eye coordinates before and after the missing data. There are
two methods for estimating the path, linear interpolation (``approx'',
the default setting) and cubic spline (``spline''). The default method
of linear interpolation replaces missing values with a line of constant
slope and evenly spaced coordinates reaching the existing data. The
cubic spline method applies piecewise cubic functions to enable a curve
to be calculated as opposed to a line between points.

When using \texttt{interpolate()}, a report can be requested so that a
researcher can measure how much missing data has been replaced. This
parameter changes the output format of the function, and returns a list
of both the data and the report. The report can be easily accessed using
the following code:

As shown, not all missing data has been replaced, since there are
certain periods in which the missing data span a period longer than the
default setting of the ``maxgap'' parameter, which is X ms. This default
setting is based on a typical duration for a blink {[}ref{]}.

Once missing data has been fixed, a common step is to smooth the eye
data to remove particularly jerky eye movements. The function
\texttt{smoother()} reduces the noise in the data by applying a moving
averaging function. The degree of smoothing can be specified, and a plot
can be generated (using data from a randomly selected trial) to observe
how well the smoothed data fits the raw data.

\subsection{Working with eyetools}\label{working-with-eyetools}

Having explained these rudimentary steps of getting the data ready for
processing with the functions in eyetools, we will now describe the core
functions available in the latest version of eyetools. For illustration,
eyetools has a built in data set that is in the required format. The
dataset consists of data from two participants from a human causal
learning study (\citeproc{ref-beesley2015}{Beesley et al., 2015}). The
nature of this experiment is largely unimportant for the current
purposes, but for clarity, the data were collected from the decision
period of the procedure, where two rectangular cue stimuli were
presented in the top half of the screen, one on the left side and one on
the right side. Two smaller response options were presented in the lower
half of the screen, one above the other. These raw data can be accessed
by calling \texttt{HCL} and the associated ``areas of interest''
(described later) can be called by using \texttt{HCL\_AOIs}.

\subsubsection{Counterbalanced designs}\label{counterbalanced-designs}

Many psychology experiments will position stimuli on the screen in a
counterbalanced fashion. For example, in the example data we are using,
there are two stimuli, with one of these appearing on the left of the
screen and the other on the right. In the design of the experiment, one
of these stimuli can be considered a ``target'' and the other a
``distractor'', and the experiment counterbalances whether these are
positioned in a left/right or a right/left arrangement across trials. In
order to provide a meaningful analysis of the eye position over all
trials, it is necessary to standardise the data such that the resulting
analyses reflect meaningful eye gaze on each type of stimulus.

\emph{eyetools} has a built in function,
\texttt{conditional\_transform()}, which allows us to \emph{transform}
the x (or y) values of the stimuli so as to take into account a
counterbalancing variable. This function currently allows for a
single-dimensional flip across either the horizontal or vertical
midline. It can be used on raw data or fixation data; we simply need to
append a column to the data to reflect the counterbalancing variable.
The result of the function is a set of data in which the x (and/or y)
position is consistent across counterbalanced conditions (e.g., in our
example, we can transform the data so that the target cue is always on
the left). This transformation is especially useful for future
visualisations and calculation of time on areas of interest. Note that
\texttt{conditional\_transform()} does not discriminate between
multi-participant and single-participant data and so no participant\_ID
parameter is required.

In our example data, the stimuli were presented on either the left or
the right side of the screen. Here we have merged the eye data with a
set of ``trial\_events'' data that describe the events on each trial. We
can apply \texttt{conditional\_transform()} and specify the relevant
column (cue\_order) that controls the counterbalancing, and the relevant
value that signals a switch of position (here ``2''). The resulting
transformation of the data will mean that the data is normalised such
that the target stimulus is always positioned on the left side of the
screen.

\subsection{Fixations}\label{fixations}

Once the data has been repaired and smoothed, a core step in eye data
analysis is to identify fixations (\citeproc{ref-salvucci2000}{Salvucci
\& Goldberg, 2000}). Broadly, a fixation is defined as a period in which
gaze stops in a specific location for a given amount of time. The period
in which the eyes are moving between fixations reflects a ``saccade''.
As such, raw data can be transformed into these meaningful eye data
characteristics. These different properties of eye-data have important
implications for behavioural research (see X for a review). Beyond their
importance for understanding psychological processes, transforming the
data into fixations and saccades leads to greater computational
efficiency. For example, the built in HCL data in eyetools is 479 kb,
which contains 31,041 rows of data (12 trials of data). After processing
the data for fixations, the resulting data is 269 rows and can be saved
as 3.8 kb (less than 1\% the size of the raw data). Not only is this
more computationally efficient, but it also means the data are more
suitable for storage in online data repositories.

There are two fixation algorithms offered in the \emph{eyetools}
package, both based on methods presented by {[}salvucci\_XXXX{]}. The
first, \texttt{fixation\_dispersion()} seeks periods of low variability
in the spatial component of the data; the algorithm looks for sufficient
periods of time in which the gaze position remains within a tolerated
maximum range of dispersion. Once this range is exceeded, this is deemed
the end of the possible period of fixation. If the total time of this
fixation period is longer than the minimum required\texttt{min\_dur}

The second algorithm, \texttt{fixation\_VTI()} takes advantage of the
idea that data is either a fixation or a saccade and employs a
velocity-threshold approach. It identifies data where the eye is moving
at a minimum velocity and excludes this, before applying a dispersion
check to ensure that the eye does not drift during the fixation period.
If the range is broken, a new fixation is determined. Saccades must be
of a given length to be removed, otherwise they are considered as
micro-saccades {[}@CITATION\_NEEDED\_HERE?{]}.

Additionally, in certain analyses it may be useful to extract the
saccades themselves. This can be achieved using the
\texttt{saccade\_VTI()} function.

Once fixations have been calculated, they can be used in conjunction
with Areas of Interest (AOIs) to determine the sequence in which the eye
enters and exits these areas, as well as the time spent in these
regions. When referring to AOIs, these often refer to the cues presented
and the outcome object. In our example, the two cues at the top of the
screen are the cues, and the outcome is at the bottom. We can define
these areas in a separate dataframe object by giving the centrepoint of
the AOI in x, y coordinates along with the width and height (if the AOIs
are rectangular) or just the radius (if circular).

In combination with the fixation data, the AOI information can be used
to determine the sequence of AOI entries using the \texttt{AOI\_seq()}
function. This fucntion checks whether a fixation is detected within an
AOI, and if not, it is dropped from the output, and then provides a list
of the sequence of AOI entries, along with start and end timestamps, and
the duration.

Time spent in AOIs can also be calculated from fixations or raw data
using the \texttt{AOI\_time()} function available. This calculates the
time spent in each AOI in each trial, based on the data type given, in
our case fixation data.

If choosing to work with the raw data, there is also the option of using
\texttt{AOI\_time\_binned()} which allows for the trials to be split
into bins of a given length, and the time spent in AOIs calculated as a
result.

\subsection{eyetool assumptions {[}I DON'T KNOW WHERE THIS SHOULD GO
JUST
YET{]}}\label{eyetool-assumptions-i-dont-know-where-this-should-go-just-yet}

As with any data processing or analysis, there are certain assumptions
made when developing the eyetools package. Some of these are built into
the package directly, either as errors or warnings, such as the
assumption that data is ordered by participant ID (if present) and
trial, and some are not built in because they would limit the
flexibility of the package functionality. One built-in assumption is the
handling of missing data. eyetools expects track loss to be represented
as NA within the data, and so any system that provides a different
convention for recording track loss needs to be changed prior to using
eyetools functions.

During development, eyetools was tested using data collected from a
Tobii Pro Spectrum eye tracker recording at 300Hz. Screen resolutions
were constant at 1080x1920 pixels, and the timestamps were recorded in
milliseconds. Whilst most of the functions were designed to work with
any hardware provided the data is formatted to eyetools expectations
(with the exception of \texttt{hdf5\_to\_df()} and
\texttt{hdf5\_get\_event()} as these convert Tobii data), as well as not
relying on specific frequencies or resolutions (either through the
function behaviour, or by supplying parameters for specificity),
eyetools has not been tested on a diverse set of datasets.

Some default behaviours are in-built, but are easily overrided such as
parameters for resolution in the plotting functions. Similarly
\texttt{saccade\_VTI()} and \texttt{fixation\_VTI()} were tested with
300Hz data. For these functions, as the frequency increases, the
relative saccadic velocities will be lower meaning that the thresholds
need to be reduced. This is important to note when working with data
that is not recorded at 300Hz. To circumvent the potential issue of
sample rates being an issue, by default functions that require a sample
rate will deduce the frequency from the data rather than needing it to
be specified.

\subsection{Visualisations made easy}\label{visualisations-made-easy}

When working with eye data, it can be beneficial for the researcher to
familiarise themselves with the dataset. Visualising the data through
graphics can help to identify meaningful patterns that are obscured when
relying on statisitical analyses alone
(\citeproc{ref-kabacoffActionThirdEdition2022}{Kabacoff, 2022}).
Graphics are also very effective at conveying information in a way that
is easily grasped by a diverse audience. eyetools offers a selection of
in-built plotting functions that work with data at most stages of
processing. These plots are designed to aid in the researcher's
processing and data analysis.

The \texttt{plot\_AOI\_growth()} function offers the representation of
how an individual (on a single trial) spends their time looking at the
different AOIs. This can be useful to see how AOIs are interacted with
over time, and this can be presented as either a cumulative over time,
or as a proportion of the time spent in the trial.

\begin{figure}[H]

\caption{\label{fig-growth}Examples of the absolute and proportional
time plots from \texttt{plot\_AOI\_growth()}}

\end{figure}%

A heatmap of eye gaze positions can be generated using
\texttt{plot\_heatmap()} which takes raw data as an input. As a
function, and unlike many of the processing steps, it does not
differentiate between trials or participants and plots any coordinate
data it is given. This behaviour is allowed as the heatmap offers an
excellent and fast ``sanity check'' that participants were, on the
whole, looking at the expected areas of the experiment screen during the
trials. As can be seen in Figure \textbf{?@fig-heatmap}, we can be
reassured that participants do indeed spend most of their time looking
at the stimuli on screen rather than in the empty space.
\texttt{plot\_heatmap()} also allows for the modification of the amount
of data displayed, using the \texttt{alpha\_control} parameter. By
decreasing \texttt{alpha\_control} in Figure
\textbf{?@fig-heatmap-alpha-update}, we gain more visualised information
and we can still see that the majority of the data is kept within the
stimuli and saccades between these areas.

The \texttt{plot\_seq()} function allows for the plotting of raw data to
visualise the gaze pattern from a single trial and where the gaze fell
on the screen across the entire trial. \textbf{?@fig-seq} offers an
example trial split into time bins of 5000ms. This plot shows the time
dimension as a change in colour that overlaps older data. This plot
serves as a useful check, similar to \texttt{plot\_heatmap()}, as to
where the eyes spent their time, but \texttt{plot\_seq()} has the
benefit of showing the time dimension compared to a simple heatmap.

The final plotting function in eyetools is \texttt{plot\_spatial()}.
This can plot raw data, fixations, and saccades, either separately or in
combination. \texttt{plot\_spatial()} plots the location of the eye gaze
of a trial, and when given raw data is very similar to the output of
\texttt{plot\_seq()}, when using fixation data, then an additional
parameter can be used to label the fixations in their temporal order,
enabling a better presentation of how fixations arise. Finally,
providing saccade data allows for the length and direction of saccades
to be presented.

\begin{figure}[H]

\caption{\label{fig-spatial}The three types of plot that can be created
using \texttt{plot\_spatial()}}

\end{figure}%

\section{Analysing eye data}\label{analysing-eye-data}

@tom

\section{Discussion}\label{discussion}

In the present tutorial, we began by identifying the current gap in
available tools for working with eye data in open-science pipelines. We
then provided an overview of the general data collection process
required for eye tracking research, before detailing the conversion of
raw eye data into a useable \emph{eyetools} format. We then covered the
entire processing pipeline using functions available in the
\emph{eyetools} package that included the repairing and normalising the
data, and the detection of events such as fixations, saccades, and AOI
entries. @SOMETHING\_ON\_THE\_ANALYSIS\_GOES\_HERE.

From a practical perspective, this tutorial offers a step-by-step
walkthrough for handling eye data using R for open-science, reproducible
purposes. It provides a pipeline that can be relied upon by novices
looking to work with eye data, as well as offering new functions and
tools for experienced researchers. By enabling the processing and
analysis of data in a single R environment it also helps to speed up
data analysis.

\subsection{Advantages of Open-Source
Tools}\label{advantages-of-open-source-tools}

eyetools offers an open-source toolset that holds no hidden nor
proprietary functionality. The major benefits of open-source tools are
extensive, but the main ones include the ability to explore and engage
with the underlying functions to ensure that

A collaborative community - with open source tools, if an unmet need is
identified, then the community can work to provide a solution.

\subsection{Good Science Practices with
eyetools}\label{good-science-practices-with-eyetools}

Creating savepoints (like having processed raw data, and then
post-fixation calculation). Reduces the need to completely rework
workflows if an issue is detected as savepoints can be used to ensure
that computationally-intense or time-heavy processes are conducted as
infrequently as possible.

\section{Data Availability}\label{data-availability}

The data required for reproducing this tutorial is available at: @URL. A
condensed version of the dataset (starting with the
\texttt{combine\_eyes()} function) is a dataset in the \emph{eyetools}
package called HCL.

\section{Code Availability}\label{code-availability}

The code used in this tutorial is available in the reproducible
manuscript file available at:(IF STORING IN GITHUB, THEN WE NEED TO
CREATE A ZENODO SNAPSHOT FOR A DOI RATHER THAN JUST A GITHUB LINK)

\section{References}\label{references}

\phantomsection\label{refs}
\begin{CSLReferences}{1}{0}
\bibitem[\citeproctext]{ref-beesley2015}
Beesley, T., Nguyen, K. P., Pearson, D., \& Le Pelley, M. E. (2015).
Uncertainty and predictiveness determine attention to cues during human
associative learning. \emph{Quarterly Journal of Experimental
Psychology}, \emph{68}(11), 2175--2199.
\url{https://doi.org/10.1080/17470218.2015.1009919}

\bibitem[\citeproctext]{ref-beesley2019}
Beesley, T., Pearson, D., \& Le Pelley, M. (2019). \emph{Chapter 1 - eye
tracking as a tool for examining cognitive processes} (G. Foster, Ed.;
pp. 1--30). Academic Press.
\url{https://doi.org/10.1016/B978-0-12-813092-6.00002-2}

\bibitem[\citeproctext]{ref-kabacoffActionThirdEdition2022}
Kabacoff, R. I. (2022). \emph{R in action: Data analysis and graphics
with r and tidyverse}. Simon; Schuster.

\bibitem[\citeproctext]{ref-salvucci2000}
Salvucci, D. D., \& Goldberg, J. H. (2000). \emph{the symposium}.
71--78. \url{https://doi.org/10.1145/355017.355028}

\end{CSLReferences}






\end{document}
